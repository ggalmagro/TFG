
\begin{titlepage}
 
 
\setlength{\centeroffset}{-0.5\oddsidemargin}
\addtolength{\centeroffset}{0.5\evensidemargin}
\thispagestyle{empty}

\noindent\hspace*{\centeroffset}\begin{minipage}{\textwidth}

\centering
%\includegraphics[width=0.9\textwidth]{imagenes/logo_ugr.jpg}\\[1.4cm]

%\textsc{ \Large PROYECTO FIN DE CARRERA\\[0.2cm]}
%\textsc{ INGENIERÍA EN INFORMÁTICA}\\[1cm]
% Upper part of the page
% 

 \vspace{3.3cm}

%si el proyecto tiene logo poner aquí
 \vspace{0.5cm}

% Title

{\Huge\bfseries Clustering Con Restricciones\\
}
\noindent\rule[-1ex]{\textwidth}{3pt}\\[3.5ex]
{\large\bfseries Un Marco Unificado.\\[4cm]}
\end{minipage}

\vspace{2.5cm}
\noindent\hspace*{\centeroffset}\begin{minipage}{\textwidth}
\centering

\textbf{Autor}\\ {Germán González Almagro}\\[2.5ex]
\textbf{Directores}\\
{Salvador García López\\
Julián Luengo Martín}\\[2cm]
%\includegraphics[width=0.15\textwidth]{imagenes/tstc.png}\\[0.1cm]
%\textsc{Departamento de Teoría de la Señal, Telemática y Comunicaciones}\\
%\textsc{---}\\
%Granada, mes de 201
\end{minipage}
%\addtolength{\textwidth}{\centeroffset}
\vspace{\stretch{2}}

 
\end{titlepage}



\cleardoublepage


%\cleardoublepage
%\thispagestyle{empty}

\begin{center}
{\large\bfseries  Clustering con Restricciones: un marco unificado}\\
\end{center}
\begin{center}
Germán González Almagro\\
\end{center}

%\vspace{0.7cm}
\noindent{\textbf{Palabras clave}: aprendizaje automático, aprendizaje semi-supervisado, clustering con restricciones, K-medias, restricciones must-link, restricciones cannot-link.}\\

\vspace{0.7cm}
\noindent{\textbf{Resumen}}\\

El clustering con restricciones a nivel de instancias es un modelo de aprendizaje automático enmarcado en el aprendizaje semi-supervisado. Permite incorporar al problema conocimiento experto sobre el dominio, de forma que los clusters resultantes del proceso de aprendizaje tengan las características deseadas. Este modelo de aprendizaje es especialmente útil y efectivo en dominios sobre los que se dispone de conocimiento profundo. En este trabajo se desarrollan las particularidades del clustering con restricciones, sus aplicaciones y sus ventajas e inconvenientes. También se exponen cinco métodos para su aplicación y se proporciona una implementación de los mismos.
\cleardoublepage


\thispagestyle{empty}


\begin{center}
{\large\bfseries Constrained Clustering: a unified framework}\\
\end{center}
\begin{center}
Germán González Almagro\\
\end{center}

\vspace{0.7cm}
\noindent{\textbf{Keywords}: machine learning, semisupervised learning, constrained clustering, K-means, must-link constraints, cannot-link constraints.}\\

\vspace{0.7cm}
\noindent{\textbf{Abstract}}\\

Clustering with instance level constraints is a machine learning model supported by the semi-supervised learning framework. The addition of constraints allows users to guide the clustering process by considering domain expertise knowledge. This model is particularly useful where considerable domain expertise exists. In this thesis we discuss constrained clustering properties and its applications, as well as its advantages and disadvantages. We also present five constrained clustering  algorithms, including their implementations.

\chapter*{}
\thispagestyle{empty}

%\noindent\rule[-1ex]{\textwidth}{2pt}\\[4.5ex]

Yo, \textbf{Germán González Almagro}, alumno de la titulación en ingeniería informática de la \textbf{Escuela Técnica Superior
de Ingenierías Informática y de Telecomunicación de la Universidad de Granada}, con DNI $\mathrm{76593910}$T, autorizo la
ubicación de la siguiente copia de mi Trabajo Fin de Grado en la biblioteca del Centro para que pueda ser
consultada por las personas que lo deseen.

\vspace{0.1cm}

\begin{flushleft}
	Granada, 17 de junio de 2018.
\end{flushleft}

\vspace{2.5cm}

\begin{flushright}
	
	\begin{tabular}{m{6cm}}
		\\ \hline
		\centering\textbf{Fdo. \myName} \\
	\end{tabular}
	
\end{flushright}


\chapter*{}
\thispagestyle{empty}

%\noindent\rule[-1ex]{\textwidth}{2pt}\\[4.5ex]

D. \textbf{Salvador García López}, Profesor del Departamento Ciencias de la Computación en Inteligencia Artificial de la Universidad de Granada.

\vspace{0.5cm}

D. \textbf{Julián Luengo Martín}, Profesor del Departamento Ciencias de la Computación en Inteligencia Artificial de la Universidad de Granada.


\vspace{0.5cm}

\textbf{Informan:}

\vspace{0.5cm}

Que el presente trabajo, titulado \textit{\textbf{<<Clustering con Restricciones: un marco unificado>>}},
ha sido realizado bajo su supervisión por \textbf{Germán González Almagro}, y autorizamos la defensa de dicho trabajo ante el tribunal
que corresponda.

\vspace{0.5cm}

Y para que conste, expiden y firman el presente informe en Granada a 17 de junio de 2018.

\vspace{1cm}

\textbf{Los directores:}

\vspace{5cm}


\begin{table}[H]
	\centering
	\begin{minipage}{.4\textwidth}
		\centering
		\begin{tabular}{c}
			\\ \hline 
			\textbf{Salvador García López}\\
		\end{tabular}
	\end{minipage}
	\hfill
	\begin{minipage}{.4\textwidth}
		\centering
		\begin{tabular}{c}
			\\ \hline 
			\textbf{Julián  Luengo Martín}\\
		\end{tabular}
	\end{minipage}
\end{table}




\chapter*{Agradecimientos}
\thispagestyle{empty}

       \vspace{1cm}


%A mis tutores, Salva y Julián, que confiaron en mí y me han guiado %en la elaboración de este trabajo.\\



\begin{flushright}
	A mis directores, Salvador y Julián \\
	\vspace{0.13cm}
	A mis amigos Javi y Tony\\
	\vspace{0.13cm}
	A mis padres
\end{flushright}
