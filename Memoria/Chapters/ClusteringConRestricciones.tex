\chapter{Clustering con restricciones}\label{ch:Clustering con restricciones}

Tal y como hemos estudiado en epígrafes anteriores, los métodos de clustering no supervisado son útiles para dotar de estructura a datos referentes a un área concreta. Un ejemplo de ello lo encontramos en la clasificación de textos; Cohn et al. (2003) afrontan un problema propuesto por Yahoo!, que consiste en, dada una gran cantidad de documentos de texto, agruparlos según una taxonomía en la que los documentos con temáticas similares se encuentren cercanos. Para ello, los métodos de clustering no supervisado resultan de utilidad, ya que la información sobre el problema de la que se dispone inicialmente es limitada. Sin embargo, Wagstaff et al. (2001) mostraron que aplicando clustering no supervisado a ciertos problemas, como el de agrupar datos de GPS de forma que los clusters definan los carriles de una vía, no se obtienen resultados significativos, pues los clusters obtenidos distan mucho de la forma alargada que se esperaría como resultado. Para atajar el problema, introdujeron en el clustering un nuevo elemento, las restricciones a nivel de instancia, que permitían incluir conocimiento sobre los clusters que guiarían los métodos de clustering para obtener los resultados esperados. Bastaba con indicar que los carriles de la vía por la que circulan los vehículos miden cuatro metros de ancho, y por tanto cualquier vehículo que se encuentre a una distancia mayor de 4 metros de otro, en dirección perpendicular a la del desplazamiento, debe ser ubicado en un cluster diferente.

Nos situamos entonces en un nuevo escenario: es posible incorporar información adicional al proceso de clustering, además de la contenida en el propio conjunto de datos, para guiarlo en la formación de la partición y obtener resultados mas precisos. Esto sitúa al clustering con restricciones en el marco del aprendizaje semisupervisado, a diferencia de los métodos de clustering tradicionales que se enmarcan en el área del clustering no supervisado. 

\section{Definición de las restricciones}

El nuevo tipo de  información que incorporamos al clustering viene dada en forma de restricciones a nivel de instancia, esto es, especificar si dos instancias del conjunto de datos deben estar en el mismo cluster, o, por el contrario, deben estar en clusters separados.

A las restricciones que indican que dos puntos deben ser situados en el mismo cluster se las denomina Must-link, y se notan por $ML(x,y)$, donde $x$ e $y$ son dos instancias del conjunto de datos. De manera similar, a las restricciones que especifican lo contrario se las denomina Cannot-link, y se notan por $CL(x,y)$.

Aunque pueden parecer simples, las restricciones definidas de la anterior forma poseen propiedades interesantes. Las Restricciones de tipo Must-link son un ejemplo de relación de equivalencia, y por tanto son simétricas, reflexivas y transitivas, formalizando: Hablando sobre

\begin{observacion}
	\textbf{Las restricciones de tipo ML son transitivas.} Sean $CC_i$ y $CC_j$ componentes conexas, conectadas mediante restricciones $ML$, y sean $x$ e $y$ dos instancias en $CC_i$ y $CC_j$ respectivamente. Entonces $ML(x,y): x \in CC_i, y \in CC_j \rightarrow ML(a,b) \forall a,b: a\in CC_i, b \in CC_j$
\end{observacion}



\begin{observacion}
	\textbf{Las restricciones de tipo CL pueden ser (encadenadas).} Sean $CC_i$ y $CC_j$ componentes conexas, conectadas mediante restricciones $ML$, y sean $x$ e $y$ dos instancias en $CC_i$ y $CC_j$ respectivamente. Entonces $CL(x,y): x \in CC_i, y \in CC_j \rightarrow CL(a,b) \forall a,b: a\in CC_i, b \in CC_j$
\end{observacion}

Un claro ejemplo de uso de las restricciones lo encontramos en casos de aplicación de clustering en los que existen restricciones en cuanto a las medidas de distancia, como sucedía en el supuesto de los datos GPS. Por ejemplo, si queremos que las instancias que forman dos clusters estén separadas por una distancia mayor o igual a $\delta$, basta con establecer restricciones de tipo ML entre todas aquellas instancias cuya distancia sea menor que $\delta$. 
De manera similar, si queremos que el diámetro de los clusters sea como mucho $\epsilon$, debemos establecer un conjunto de restricciones de tipo CL entre todas aquellas instancias que se encuentren a una distancia mayor que $\epsilon$. La figura \ref{fig:figure4} muestra una representación gráfica de estos dos tipos de restricciones.

\begin{figure}[!h]
	\centering
	\includegraphics[scale=0.45]{imagenes/c3/RestriccionesDeltaEpsilo.png} 
	\caption{Restricciones de tipo delta y epsilon. \cite{Survey:2007}}\label{fig:figure4}
\end{figure}

\clearpage

\section{Uso de las restricciones}

Mientras que el aprendizaje completamente supervisado implica conocer la etiqueta asociada a cada instancia, en el aprendizaje semisupervisado solo se dispone de un subconjunto de instancias etiquetadas. Por otra parte, en gran cantidad de dominios la información disponible se refiere a relaciones entre instancias, y no a la clase concreta a la que pertenecen las mismas. Es más, en montajes de clustering interactivo, un usuario no experto en el dominio del problema podrá, probablemente, aportar información en forma de restricciones de tipo \acf{ML} y \acf{CL}, antes que aportar información sobre a qué clase concreta pertenecen ciertas instancias.

Habitualmente, las restricciones se incluyen en los problemas de clustering de dos maneras. Pueden ser empleadas para modificar la regla de asignación de instancias a cluster del método en cuestión, de forma que la solución satisfaga el máximo número de restricciones posible. Alternativamente, cabe la posibilidad de entrenar la función de distancia empleada por el método en base a las restricciones, ya sea antes o durante la aplicación del mismo. En cualquier caso, la fase incialización puede tomar en consideración las restricciones, de forma que las instancias asociadas con restricciones \acf{ML} serán situadas en el mismo cluster, y aquellas entre las que exista una restricción \acf{CL}, quedarán en clusters diferentes. Basándonos en esta distinción, identificamos dos maneras de aproximar el problema, las basadas en restricciones (\textit{constraint-based}), y las basadas en distancias (\textit{distance-based}).

\subsection{Métodos basados en restricciones}

En los métodos basados en restricciones, el propio método de clustering es modificado de manera que la información disponible se emplea para sesgar la búsqueda y obtener una partición de los datos apropiada.

Existen dos modelos de métodos basados en restricciones: (1) aquellos que fuerzan el cumplimiento de las restricciones, e intentan encontrar la mejor asignación posible que no inflija ninguna de ellas, y (2) las que hacen una interpretación relajada de las restricciones, permitiéndose incumplir un número mínimo de ellas para optimizar la función objetivo, de esta manera surge un compromiso entre el número de restricciones incumplidas y el valor de la función objetivo. Este tipo de métodos emplean diversas técnicas para lograr obtener una partición atendiendo a las restricciones:

\begin{itemize}
	
	\item Modificar la función objetivo de manera que incluya una penalización por incumplir restricciones.
	
	\item Agrupar con información adicional obtenida de una distribución condicional en un espacio auxiliar.
	
	\item Forzar el cumplimiento de todas las restricciones modificando la regla de asignación del método.
	
	\item Inicializando los clusters e base a restricciones inferidas del conjunto de instancias etiquetadas.
	
\end{itemize}

La figura \ref{fig:figure5} muestra un conjunto de datos junto a sus restricciones asociadas, la \ref{fig:figure6} propone un posible agrupamiento que satisface todas las restricciones.

\begin{figure}[bth]
	\myfloatalign
	{\includegraphics[width=.6\linewidth]{imagenes/c3/InputInstancesAndConst1}
	\caption{Restricciones sobre un conjunto de datos \cite{Survey:2007}} \label{fig:figure5}
	}
	{\includegraphics[width=.6\linewidth]{imagenes/c3/ClusteringSatAll}
	\caption{Clustering que satisface todas las restricciones \cite{Survey:2007}} \label{fig:figure6}
	}
\end{figure}

\subsection{Métodos basados en distancia}

En las aproximaciones basadas en distancias, se emplean métodos de clustering clásicos que hagan uso de una medida de distancia, de forma que dicha medida se modifica para que tenga en consideración las restricciones. En este contexto, satisfacer las restricciones significa que las instancias relacionadas con restricciones \acf{ML} se sitúan juntas en el espacio, y las relacionadas mediante \acf{CL} se encuentran separadas.

La figura 3.5 muestra un posible agrupamiento basado en una métrica aprendida a partir de las restricciones especificadas en la figura \ref{fig:figure7} Cabe destacar que en la figura \ref{fig:figure8} el espacio en el que se encuentran los datos ha sido comprimido en el eje vertical y ensanchado en el eje horizontal para ajustarlo a la métrica de distancia aprendida.

\begin{figure}[bth]
	\myfloatalign
	{\includegraphics[width=.6\linewidth]{imagenes/c3/InputInstancesAndConst2}
	\caption{Restricciones sobre un conjunto de datos \cite{Survey:2007}} \label{fig:figure7}
	}
	{\includegraphics[width=.6\linewidth]{imagenes/c3/MetricaAprendida}
	\caption{Clustering basado en métrica aprendida en base a las restricciones \cite{Survey:2007}} \label{fig:figure8}
	}
\end{figure}

\section{Aplicaciones del clustering con restricciones} 

Este epígrafe muestra algunos casos de aplicación en los que el clustering con restricciones ha resultado ser una herramienta más útil que el clustering no supervisado. Para cada caso analizaremos como se obtuvieron las restricciones y como estas mejoran los resultados en el clustering resultante. 

\subsection{Aplicaciones en análisis de imágenes}

La figura \ref{fig:figure9} muestra un extracto del conjunto de datos de caras de  \acf{CMU}, en el que la tarea es agrupar caras en base a diferentes criterios. En este caso, el objetivo es agrupar las caras según su orientación, es decir, las caras con la misma orientación deberán estar en el mismo cluster.

\begin{figure}[bth]
	\myfloatalign
	\subfloat[Caras de perfil]
	{\includegraphics[width=.3\linewidth]{imagenes/c3/AnalisisImagenes/Caras1}} \quad
	\subfloat[Caras de frente.]
	{\includegraphics[width=.3\linewidth]{imagenes/c3/AnalisisImagenes/Caras3}} \quad
	\subfloat[Caras hacia arriba.]
	{\includegraphics[width=.3\linewidth]{imagenes/c3/AnalisisImagenes/Caras2}} \quad
	\caption{Caras de la base de datos de  \ac{CMU} \cite{Survey:2007}}\label{fig:figure9}
\end{figure}

El método empleado para extraer las restricciones es uno de los más populares en la literatura: establecer el número de clusters igual al número de clases en la base de datos, y generar las restricciones a partir de un subconjunto de instancias etiquetadas, esto es, si dos instancias tiene diferentes etiquetas, establecer una restriccion \acf{CL} entre ellas, en caso contrario una de tipo \acf{ML}. De esta forma, entre las imágenes mostradas en la figura \ref{fig:figure10} se establecen restricciones \acf{CL}, ya que, aunque pertenecen a la misma persona, no presentan la misma orientación.

\begin{figure}[bth]
	\myfloatalign
	{\includegraphics[width=.4\linewidth]{imagenes/c3/AnalisisImagenes/CarasDifOr1}} \quad
	{\includegraphics[width=.4\linewidth]{imagenes/c3/AnalisisImagenes/CarasDifOr2}}
	\caption{Restricciones de tipo \ac{CL} entre caras de la misma persona  \cite{Survey:2007}}\label{fig:figure10}
\end{figure}

En la figura \ref{fig:figure11} se muestra otro conjunto de datos de imágenes sobre el que se aplican técnicas de clustering con restricciones. En este caso, la tarea es realizar reconocimiento de objetos para incorporar el método al sistema de navegación del robot Aibo. Para ello se emplean restricciones de distancia de tipo $\delta$ y $\epsilon$ como las descritas en la figura \ref{fig:figure4}, de esta manera se consiguen clusters bien diferenciados y por tanto útiles para las tareas de búsqueda de caminos que el robot realiza durante la navegación.

\begin{figure}[bth]
	\myfloatalign
	\subfloat[Imagen original]
	{\includegraphics[width=.4\linewidth]{imagenes/c3/AnalisisImagenes/Aibo1}} \quad
	\subfloat[Clustering sin restricciones]
	{\includegraphics[width=.4\linewidth]{imagenes/c3/AnalisisImagenes/Aibo2}} \quad
	\subfloat[Clustering con restricciones]
	{\includegraphics[width=.45\linewidth]{imagenes/c3/AnalisisImagenes/Aibo3}} \quad
	\caption{Método de clustering empleado en el sistema de navegación del robot Aibo \cite{Survey:2007}}\label{fig:figure11}
\end{figure}

\subsection{Aplicaciones en análisis de vídeos}

Las bases de datos de video son uno de los ejemplos en los que las restricciones pueden ser generadas directamente desde el dominio de datos, especialmente disponiendo de datos espacio-temporales sobre el vídeo. En datos temporalmente sucesivos es posible establecer restricciones de tipo \acf{ML} entre grupos de píxeles de fotogramas (\textit{frames}) cercanos en el tiempo. Esto es especialmente útil cuando la tarea es implementar reconocimiento de objetos basado en clustering y segmentación. También es posible añadir restricciones \acf{CL} a clusters localizados en el mismo fotograma, ya que existe una baja probabilidad de que estén asociados al mismo objeto. De hecho, en el dominio asociado a problemas de análisis de vídeo existen gran variedad métodos de extracción de restricciones, la figura \ref{fig:figure12} muestra algunos ejemplos.  

\begin{figure}[bth]
	\myfloatalign
	\subfloat[]
	{\includegraphics[width=.4\linewidth]{imagenes/c3/Videos/VideoA}}
	\quad
	\subfloat[]
	{\includegraphics[width=.225\linewidth]{imagenes/c3/Videos/VideoB}} \quad
	\subfloat[]
	{\includegraphics[width=.4\linewidth]{imagenes/c3/Videos/VideoC}}
	\quad
	\subfloat[]
	{\includegraphics[width=.4\linewidth]{imagenes/c3/Videos/VideoD}}
	\caption{Diferentes tipos de restricciones en datos de video \cite{Survey:2007}}\label{fig:figure12}
\end{figure}

En la figura \ref{fig:figure12}, la imagen (a) corresponde a restricciones extraídas del seguimiento de una persona durante un periodo de tiempo, la (b) corresponde a restricciones espaciales que asocian dos objetos localizados en el mismo fotograma, la imagen (c) corresponde a restricciones obtenidas mediante reconocimiento facial y la (d) a las proporcionadas por el usuario.

Disponiendo de tantos métodos para extraer restricciones, una cuestión que cabe plantearse en este contexto es, ¿qué sucede si se establecen demasiadas restricciones? ¿Hace esto que el problema esté sobrerestringido? En epígrafes posteriores abordaremos estas cuestiones.

\subsection{Aplicaciones en genética}

En clustering de genes basado en microarrays, los genes viene representados por si perfil de expresión en diferentes experimentos y agrupados empleando diferentes algoritmos, en este caso algoritmos de clustering con restricciones. La  figura \ref{fig:figure13} muestra un ejemplo, en este caso las restricciones de tipo \acf{ML} se establecen entre genes en base a los datos de co-ocurrencia almacenados en la base de datos de interacciones de proteínas, que contiene información sobre qué genes (y sus proteínas asociadas) están asociados a los mismos procesos celulares.

\begin{figure}[!h]
	\centering
	\includegraphics[scale=0.3]{imagenes/c3/Genetica/Genes} 
	\caption{Clustering de genes basado en microarrays \cite{Survey:2007}}\label{fig:figure13}
\end{figure}

\subsection{Aplicaciones en análisis de textos}




























