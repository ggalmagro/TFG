%Definiciones de español para el paquete algorithm2e
\ifthenelse{\boolean{algocf@optonelanguage}\AND\equal{\algocf@languagechoosen}{spanish}}{%
	\SetKwInput{KwIn}{Entrada}%
	\SetKwInput{KwOut}{Salida}%
	\SetKwInput{KwData}{Datos}%
	\SetKwInput{KwResult}{Resultado}%
	\SetKw{KwTo}{a}%
	\SetKw{KwRet}{devolver}%
	\SetKw{Return}{devolver}%
	\SetKwBlock{Begin}{inicio}{fin}%
	\SetKwRepeat{Repeat}{repetir}{hasta que}%
	%
	\SetKwIF{If}{ElseIf}{Else}{si}{entonces}{sin\'o, si}{en otro caso}{fin si}
	\SetKwSwitch{Switch}{Case}{Other}{seleccionar}{hacer}{caso}{sin\'o}{fin caso}{fin seleccionar}
	\SetKwFor{For}{per}{fai}{fine per}%
	\SetKwFor{ForPar}{par}{hacer in paralelo}{fin para}%
	\SetKwFor{ForEach}{para cada}{hacer}{fin para cada}
	\SetKwFor{ForAll}{para todo}{hacer}{fin para todo}
	\SetKwFor{While}{mientras}{hacer}{fin mientras}
}{}%

\section{Motivación Personal}

Durante mi formación he escuchado de multitud de profesores que incorporar conocimiento humano a una máquina es una de las tareas más complejas a las que se ha enfrentado la humanidad.

Algo que realmente me resultó interesante fue que Deep Blue, la primera máquina en ganar al campeón del mundo de ajedrez, Gary Kasparov en aquel momento (1996), no ganó por un avance significativo en el algoritmo que ejecutaba la máquina, sino por avances en el hardware, que permitían que la máquina analizase más movimientos por unidad de tiempo, es decir, la máquina no empleaba conocimiento del que no disponía anteriormente, simplemente ``pensaba'' más rápido.

Por ello, he querido profundizar en el campo de la incorporación de conocimiento a las máquinas algo más de lo que he podido hacerlo durante estos años. El clustering con restricciones puede verse como un ejemplo de ello, al fin y al cabo no es más que guiar el proceso de toma de decisiones de una máquina incorporando conocimiento extraído de las personas.
